%        File: netsukuku.tex
%     Created: Sun Nov 19 03:00 PM 2006 C
% Last Change: Sun Nov 19 03:00 PM 2006 C
%
% This file is part of Netsukuku
% (c) Copyright 2007 Andrea Lo Pumo aka AlpT <alpt@freaknet.org>
%
% This source code is free software; you can redistribute it and/or
% modify it under the terms of the GNU General Public License as published 
% by the Free Software Foundation; either version 2 of the License,
% or (at your option) any later version.
%
% This source code is distributed in the hope that it will be useful,
% but WITHOUT ANY WARRANTY; without even the implied warranty of
% MERCHANTABILITY or FITNESS FOR A PARTICULAR PURPOSE.
% Please refer to the GNU Public License for more details.
%
% You should have received a copy of the GNU Public License along with
% this source code; if not, write to:
% Free Software Foundation, Inc., 675 Mass Ave, Cambridge, MA 02139, USA.
%

\documentclass[a4paper]{article}
\usepackage{color,graphicx}
\usepackage{amsmath}
\usepackage{amsthm}
\usepackage{amssymb}
\usepackage{amsfonts}
\RequirePackage{ifpdf} % running on pdfTeX?
\ifpdf
\usepackage[pdftex,bookmarks=true,
		   bookmarksnumbered=false,
		   bookmarksopen=false,
		   colorlinks=true,
		   linkcolor=webred] {hyperref}
\definecolor{webgreen}{rgb}{0, 0.5, 0} % less intense green
\definecolor{webblue}{rgb}{0, 0, 0.5} % less intense blue
\definecolor{webred}{rgb}{0.5, 0, 0}   % less intense red
\else
\newcommand{\href}[2]{ #1 }
\fi
\title{Netsukuku\\
{\small Close the world, \reflectbox{Open the next}}}
\author{http://netsukuku.freaknet.org}
	%AlpT (@freaknet.org)
\begin{document}
\maketitle

\begin{abstract}
Netsukuku e' un sistema di rete p2p creato per gestire un numero pressoche' 
illimitato di nodi con minime risorse di CPU e memoria. E' idoneo a far 
nascere agevolmente una rete mondiale distribuita, anonima e non controllabile, 
totalmente avulsa da Internet, dai suoi server, ISP ed Enti di controllo.\\
In questo documento, diamo una descrizione generica e non tecnica della rete 
Netsukuku, evidenziandone l'idea di fondo e le principali caratteristiche.
\end{abstract}


\section{La vecchia Rete}

Internet e' una rete a struttura gerarchica gestita da societa' multinazionali e da organizzazioni governative. Non esiste un solo byte di traffico su Internet che non passi attraverso le dorsali ed i router delle societa' di telecomunicazioni che ne sono proprietari.\\
Gli Internet Service Provider (ISP) offrono connettivita' a tutti gli utenti che si trovano nel livello piu' basso di una struttura rigorosamente gerarchica e piramidale. Per questa ragione, Internet non puo' essere affatto considerata una rete globale condivisa tra i suoi utenti. Le persone possono partecipare a questa grande rete solo se s'impegnano ad accettare le condizioni ed i termini rigorosamente �imposti dalle multinazionali.
Internet rappresenta, oggi, la rete per eccellenza di accesso
all'informazione, al sapere e alla comunicazione. Circa un miliardo di persone
puo' connettersi a questa enorme rete proprietaria, mentre ai rimanenti cinque
miliardi di abitanti, che vivono in Paesi con risorse insufficienti, non
rimane altro che attendere che le multinazionali delle telecomunicazioni
decidano come e quando offrire loro connettivita'.
Internet e' stata in origine concepita col fine di garantire comunicazioni
sicure ed inattaccabili tra diversi nodi della rete, ma oggi, paradossalmente,
un qualsiasi ISP ha il potere di tagliare fuori da Internet intere nazioni,
semplicemente interrompendo la fornitura del proprio servizio.
Internet, inoltre, non e' ne potra' mai essere anonima: gli ISP posso
rintracciare ed analizzare il traffico di dati che passano attraverso i loro
server, senza alcuna limitazione.
Internet da' origine, come naturale conseguenza, ad altri sistemi improntati
alla sua stessa natura gerarchica e centralizzata, come, ad esempio, il DNS. I
server del Domain Name System sono gestiti da diversi ISP, e gli stessi domini sono 
letteralmente venduti attraverso un simile sistema centralizzato. Questo
insieme di servizi consente, in modo molto semplice ed efficace, di
localizzare fisicamente qualunque computer collegato a Internet, in un tempo
molto breve e senza alcun particolare sforzo.
In Cina, l'intera rete e' costantemente controllata da diversi computer che
filtrano il traffico Internet: un cittadino cinese non sara' mai in grado di
visionare o venire a conoscenza di siti che contengono parole chiavi che, come
``democrazia'', sono sottoposte a censura dal suo governo. Non gli sara' mai concessa la possibilita' di esprimere le proprie idee politiche, senza, per questo, rischiare persino una condanna alla pena di morte.
D'altronde, Internet e' nata con il fine di soddisfare i bisogni di sicurezza
dell'amministrazione militare degli stati Uniti d'America e non certo per
assicurare la liberta' di comunicazione e d'informazione. Ogni utente di
Internet, oggi, per poter comunicare con altri utenti, e' obbligato a
sottomettersi al controllo e ricorrere ai servizi delle grandi multinazionali, la cui
vera missione e' di espandere sempre piu' la propria egemonia.\\
Essendo un dato di fatto che tutti gli sforzi per apportare su Internet
maggiore liberta', tutela della privicy e garanzie di accessibilita' si
scontrano con l'avversione, le fobie e gli interessi contrari di Governi e
compagnie private, la soluzione piu' adeguata al problema e' quella di offrire
a tutti l'alternativa di migrare verso un'altra rete totalmente avulsa da
Internet, che sia efficiente, distribuita e decentralizzata, in modo tale che
al suo interno ci siano soltanto utenti che interagiscano tra loro in maniera
assolutamente paritetica, senza privilegi, sistemi di condizionamento o di
controllo, per dar modo a chiunque vi sia connesso di far parte di una nuova
comunita' mondiale veramente libera.

\section{The Netsukuku wired}
Netsukuku e' una mesh network o sistema di rete p2p che si 
genera e sostiene autonomamente. E' stato concepito per supportare
un numero pressoche' illimitato di nodi, richiedendo dalla CPU e 
dalla memoria di ciascun PC collegato un impiego minimale di risorse.
Grazie alle sue caratteristiche Netsukuku 
puo' essere facilmente impiegato per costruire una rete non controllata, 
anonima e distribuita, totalmente staccata da Internet, dai suoi server, 
ISP e dai suoi enti di governo.\\
Questa rete e' costituita dagli stessi computer collegati fisicamente 
l'un l'altro e, pertanto, non poggia su Internet o su alcun altra rete esistente.\\
Netsukuku provvede autonomamente a stabilire le rotte tra i computer: 
in termini tecnici, Netsukuku rimpiazza il livello 3 del modello iso/osi 
con un altro protocollo di routing.\\
Poiche' Netsukuku e' una rete a struttura distribuita e decentralizzata, 
rende possibile l'implementazione di altri sistemi distribuiti poggianti 
su di essa come, ad esempio, l'Abnormal Netsukuku Domain Name Anarchy 
(ANDNA) \cite{andnadoc} che rimpiazza il sistema DNS gerarchico e 
centralizzato attualmente in uso su Internet.

\subsection{Gandhi}
La principale caratteristica di Netsukuku e' quella dell'auto-sostentamento: 
la rete si configura dinamicamente, senza mai aver necessita' di alcun 
intervento esterno. Tutti i nodi condividono gli stessi privilegi e limitazioni, 
dando lo stesso contributo al sostentamento ed all'espansione della rete Netsukuku. 
Piu' i nodi crescono in numero e piu' la rete si espande e diventa efficiente.\\
La totale decentralizzazione e distribuzione consente a Netsukuku di non essere 
ne' controllabile ne' distruggibile: l'unico modo per manipolare o demolirla e' di 
abbattere fisicamente ogni singolo nodo di cui la rete e' composta.


\subsection{Nessun nome, nessuna identita'}

All'interno della rete Netsukuku chiunque, in qualunque luogo, in ogni momento puo' 
agganciarsi immediatamente alla rete, senza passare attraverso procedure burocratiche 
o adempimenti legali.\\
Ogni elemento della rete e' estremamente dinamico non rimanendo mai identificato 
univocamente. Infatti, l'indirizzo IP che identifica ciascun computer e' assegnato 
casualmente ed e' estremamente difficile associarlo a un determinato luogo fisico.
Inoltre, poiche' le rotte attraversano un alto numero di nodi, e' un'impresa 
davvero titanica quella di rintracciare uno specifico nodo.\\
Il traffico dei nodi e' protetto da un layer di crittografia completo \cite{carciofo}, 
che assicura un grado elevato di anonimita' e sicurezza.


\subsection{Allora, cos'e'?}

Netsukuku e' una mesh network che si sostiene grazie al proprio protocollo di routing 
dinamico. 
Attualmente esiste un largo numero di protocolli di routing dinamico, ma sono 
utilizzabili soltanto per creare reti di piccole o medie dimensioni. Gli stessi routers 
di Internet sono gestiti con diversi protocolli come l'OSPF, il RIP o il BGP, fondati 
su differenti algoritmi classici della teoria dei grafi, che sono capaci di trovare la rotta
migliore per raggiungere un nodo all'interno di un dato grafo. Tuttavia, tutti 
questi protocolli richiedono un impiego massiccio di CPU e di memoria. Per
questo motivo, 
i router di Internet sono computer dedicati espressamente all'esecuzione dei predetti 
algoritmi. Sarebbe impossibile implementare uno solo di questi protocolli per creare 
e mantenere una mesh network con un numero di utenti simile a Internet.
Il protocollo di Netsukuku struttura l'intera rete come un frattale\cite{ntktopology} 
e, per calcolare tutte le rotte necessarie, fa uso di un particolare algoritmo denominato 
Quantum Shortest Path Netsukuku \cite{qspndoc}.
Un frattale e' una struttura che puo' essere compressa all'infinito, perche' ogni sua 
parte e' composta dallo stesso frattale. Pertanto il suo alto livello di compressibilita' 
offre la possibilita' di conservare l'intera mappa di Netsukuku in soli pochi Kilobyte.
D'altra parte, il QSPN e' un algoritmo che ha il compito di essere eseguito dalla stessa rete. 
I nodi lo eseguono semplicemente inviando e ricevendo i Tracer Packets, senza impegnare 
affatto pesanti risorse computazionali.

\subsection{Netsukuku e il wireless}
Il modo migliore per stabilire connessioni fisiche tra nodi e' il wifi.
Quando Netsukuku sara' ampiamente diffuso, i suoi utenti dovranno soltanto
piazzare la propria antenna wifi in maniera ben
esposta (ad esempio finestre o tetti) e connettersi ad altri utenti di
Netsukuku situati all'interno della loro portata radio.
Ad oggi, esiste una grande varieta' di tecnologie wifi che consentono di
collegare due nodi distanti anche chilometri tra loro. Inoltre, un'intera
citta' puo' essere facilmente coperta perche' la sua alta densita' abitativa
rende necessario solo un nodo-ponte per ogni quartiere.\\
Tra l'altro e' possibile utilizzare tunnel virtuali poggianti su Internet per
rimpiazzare temporaneamente i link fisici mancanti. Vedi \cite{inetdoc}.

%%%%%%%%%%%%%%%%
% Bibliography %
%%%%%%%%%%%%%%%%

\begin{thebibliography}{99}
	\bibitem{ntksite} Netsukuku website:
		\href{http://netsukuku.freaknet.org/}{http://netsukuku.freaknet.org/}
	\bibitem{qspndoc} Documento del QSPN:
		\href{http://netsukuku.freaknet.org/doc/main\_doc/qspn.pdf}{qspn.pdf}
	\bibitem{ntktopology} Netsukuku topology document:
		\href{http://netsukuku.freaknet.org/doc/main\_doc/topology.pdf}{topology.pdf}
	\bibitem{andnadoc} Documento dell'ANDNA:
		\href{http://netsukuku.freaknet.org/doc/main\_doc/andna.pdf}{andna.pdf}
	\bibitem{carciofo} Carciofo NTK\_RFC:
		\href{http://lab.dyne.org/Ntk\_carciofo}{Carciofo}
	\bibitem{inetdoc}  Internet e Netsukuku:
		\href{http://netsukuku.freaknet.org/doc/main\_doc/inetntk.pdf}{inetntk.pdf}
\end{thebibliography}
\newpage

\begin{center}
\verb|^_^|
\end{center}
\end{document}
