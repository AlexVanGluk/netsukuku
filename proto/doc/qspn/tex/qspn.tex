%        File: qspn.tex
%     Created: Fri Oct 13 09:00 PM 2006 C
% Last Change: Fri Oct 13 09:00 PM 2006 C
%
\documentclass[a4paper]{article}
\usepackage{color,graphicx}
\usepackage{amsmath}
\usepackage{amsthm}
\usepackage{amssymb}
\usepackage{amsfonts}
\RequirePackage{ifpdf} % running on pdfTeX?
\ifpdf
\usepackage[pdftex]{hyperref}
\else
\newcommand{\href}[2]{ #1 }
\fi
\title{Quantum Shortest Path Netsukuku}
\author{AlpT (@freaknet.org)}
\begin{document}
\maketitle

\begin{abstract}
	This document describes the QSPN, the routing discovery algorithm used
	by Netsukuku.
	Through a deductive analysis the main proprieties of the QSPN are
	shown. Moreover, a second version of the algorithm, is presented.
\end{abstract}

\section{Preface}
\label{sec:preface}

The first part of the document describes the reasoning which led us to the
construction of the current form of the QSPN v2.
If you are just interested in the description of the QSPN v1 and v2, you can
directly skip to section \ref{sec:TPC}.

\section{The general idea}
\label{sec:general_idea}

The aim of Netsukuku is to be a (physical) scalable mesh network, completely
distributed and decentralised, anonymous and autonomous.

The software, which must be executed by every node of the net, has to be
unobtrusive. It has to use very few CPU and memory resources, in this way it
will be possible to run it inside low-performance computers, like Access Points,
embedded devices and old computers.

If this requirements are met, Netsukuku can be easily used to build a worldwide
distributed, anonymous and not controlled network, separated from the
Internet, without the support of any servers, ISPs or authority controls.

\subsection{The network model}
\label{sec:net_model}

Netsukuku prioritises the stability and the scalability of net: the network
has to be able to grow to even $2^{2^7}$ nodes.
For this reason, it's supposed that a node won't change its physical location
quickly nor often.

Some consequences of this assumption are: 
\begin{enumerate}
	\item	Mobiles node aren't supported by Netsukuku algorithms.
		\footnote{It is possible to use other mesh network protocols
		designed for mobility in conjunction with Netsukuku (f.e. see
		\href{http://olsrd.org}{olsrd}), in the same way they are used in conjunction
		with the Internet.}
	\item   The network isn't updated quickly: several minutes may be
		required before all the nodes become aware of a change of the
		network (new nodes have joined, more efficient routes have
		become available, \dots). However, when a node joins
		the network, it can reach all the other nodes from the first
		instant, using the routes of its neighbours.
\end{enumerate}


\subsection{The routing algorithm}
One of the most important part of Netsukuku, is the routing discovery
algorithm, which is responsible to find all the most efficient routes of the
network. These routes will permit to each node to reach any other node.

The routing algorithm must be capable to find the routes without overloading
the network or the nodes' CPU and memory resources.

\subsection{The QSPN}

Netsukuku implements its own algorithm, the \emph{QSPN} (\textbf{Q}uantum
\textbf{S}hortest \textbf{P}ath \textbf{N}etsukuku). The name derives from the
way of working of its principal component: the \emph{TP} (Tracer Packet), a
packet which gains a ``quantum'' of information at each hop.

The QSPN is based on the assumptions made in section \ref{sec:net_model}.

\section{Network topology}
\label{sec:net_topology}

The QSPN alone wouldn't be capable of handling the whole network, because it
would still require too much memory. For example, even if we store just one
route to reach one node and even if this route costs one byte, we would need
1Gb of memory for a network composed by $10^9$ nodes (the current Internet).

For this reason, it's necessary to structure the network in a convenient
topology.

\subsection{Fractal topology}
Netsukuku, adopts a fractal like structure:
256 nodes are grouped inside a \emph{group node} (gnode), 256 group nodes are grouped
in a single \emph{group of group nodes} (ggnode), 256 group of group nodes are
grouped in a gggnode, and so on.
(We won't analyse the topology of Netsukuku. You can find more informations
about it in the main documentation: \cite{ntksite}).
\newline
Since each gnode acts as a single real node,
the QSPN is able to operate independently on each level of the fractal.

Because in each level there are a maximum of 256 (g)nodes, the QSPN will
always operate on a maximum of 256 (g)nodes, therefore we would need just to
be sure that it works as aspected on every cases of a graph composed by $\le
256$ nodes. By the way, we'll directly analyse the general case.

For the sake of simplicity, in this paper, we will assume to operate on level
0 (the level formed by 256 single nodes).

\section{Tracer packet}
\label{sec:TP}

A \emph{TP} (Tracer Packet) is the fundamental concept on which the QSPN is
based: 
it is a packet which stores in its body the IDs of the traversed hops.

\subsection{Tracer packet flood}

A TP isn't sent to a specific destination but instead, it is used to flood the
network. By saying ``the node A sends a TP'' we mean that ``the node A is
starting a TP flood''.

% TODO: continue here

\subsection{Proprieties of the tracer packet}
\begin{enumerate}
	\item A node $D$ which received a TP, can know the exact route covered
		by the TP. Therefore, $D$ can know the route to reach the
		source node $S$, which sent the TP, and the routes to reach
		the nodes standing in the middle of the route.
		
		For example, suppose that the TP received by $D$ is: $\left\{
		S, A, B, C, D \right\}$. By looking at the packet $D$ will
		know that the route to reach $B$ is $C\rightarrow B$, to reach $A$ is
		$C\rightarrow B\rightarrow A$, and finally to reach $S$ is
		$C\rightarrow B\rightarrow A\rightarrow S$.
		The same also applies for all the other nodes which received
		the TP, f.e, $B$ knows that its route to reach $S$ is
		$A\rightarrow S$.
\end{enumerate}


\subsection{Example}
\begin{figure}[h]
	\begin{center}
		\includegraphics[scale=0.4]{fig/segABCDEF}
	\end{center}
	\caption{A simple graph}
\end{figure}

Suppose that $D$ sends a TP:



\section{Raw flood}
Ogni nodo manda un TP

\section{Routes simplification}

\section{Cyclic tracer packet}
\label{sec:TPC}

\begin{thebibliography}{99}
	\bibitem{ntksite} Netsukuku website:
		\href{http://netsukuku.freaknet.org/}{http://netsukuku.freaknet.org/}.
\end{thebibliography}

\end{document}
