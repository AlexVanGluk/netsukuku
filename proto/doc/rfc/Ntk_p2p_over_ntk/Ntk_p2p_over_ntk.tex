%        File: Ntk_p2p_over_ntk.tex
%     Created: Thu Aug 02 07:00 PM 2007 C
% Last Change: Thu Aug 02 07:00 PM 2007 C
%
% This file is part of Netsukuku
% (c) Copyright 2007 Andrea Lo Pumo aka AlpT <alpt@freaknet.org>
%
% This source code is free software; you can redistribute it and/or
% modify it under the terms of the GNU General Public License as published 
% by the Free Software Foundation; either version 2 of the License,
% or (at your option) any later version.
%
% This source code is distributed in the hope that it will be useful,
% but WITHOUT ANY WARRANTY; without even the implied warranty of
% MERCHANTABILITY or FITNESS FOR A PARTICULAR PURPOSE.
% Please refer to the GNU Public License for more details.
%
% You should have received a copy of the GNU Public License along with
% this source code; if not, write to:
% Free Software Foundation, Inc., 675 Mass Ave, Cambridge, MA 02139, USA.
%

\documentclass[a4paper]{article}
\usepackage{color,graphicx}
\usepackage{amsmath}
\usepackage{amsthm}
\usepackage{amssymb}
\usepackage{amsfonts}
\RequirePackage{ifpdf} % running on pdfTeX?
\ifpdf
\usepackage[pdftex,bookmarks=true,
		   bookmarksnumbered=false,
		   bookmarksopen=false,
		   colorlinks=true,
		   linkcolor=webred] {hyperref}
\definecolor{webgreen}{rgb}{0, 0.5, 0} % less intense green
\definecolor{webblue}{rgb}{0, 0, 0.5} % less intense blue
\definecolor{webred}{rgb}{0.5, 0, 0}   % less intense red
\else
\newcommand{\href}[2]{ #1 }
\fi

%%%% Misc
\newcommand{\T}[1]{\textrm{#1}}
\newcommand{\see}[1]{\T{[\ref{#1},pg.\pageref{#1}]}}
\newcommand{\vedi}[1]{\T{vedi \see{#1}}}
\newcommand{\pgra}[1]{\left\{#1\right\}}
\newcommand{\pqua}[1]{\left[#1\right]}
\newcommand{\pton}[1]{\left(#1\right)}
\newcommand{\pass}[1]{\Big|#1\Big|}
\newcommand{\sist}[1]{{ \begin{cases} #1 \end{cases} }}
\newcommand{\eal}[1]{{\begin{align*} #1 \end{align*}}}
\def\ove#1{{\overline{#1}}}
\newcommand{\qq}{\qquad}
%% Defs
\def\*{{\times}}
\def\|{{\;\lor\;}}
\def\&{{\;\land\;}}
\def\-{{\setminus}}
\def\0{{\emptyset}}
\def\8{{\infty}}
\def\v{{\cup}}
\def\^{{\cap}}
\def\<{{\;\Leftarrow\;}}
\def\>{{\;\Rightarrow\;}}
\def\={{\;\Leftrightarrow\;}}
\def\({{\subseteq}}
\def\){{\supseteq}}
\def\'{{\;\;\;}}
\def\,{{,\;}}

\theoremstyle{definition}
\newtheorem{defn}{Definition}[section]

\title{P2P over Netsukuku\\\small{NTK RFC 0014}}
\author{http://netsukuku.freaknet.org}
\begin{document}
\maketitle
\begin{abstract}
This text describes how is it possible to create a distributed P2P service
over the Netsukuku network. As example, a distributed P2P Bittorrent service
is presented.
\end{abstract}

\section{Introduction}
Netsukuku is a distributed, collaborative network of nodes. 
For this reason, the development of P2P applications over Netsukuku is
rather easy.

A P2P application over Ntk can directly access the information
regarding every part of the network by reading the maps and can known
immediately its dynamic changes by listening to QSPN packets.
In order to ease the development of such applications, a ``ntkp2p'' library will
be developed.

\section{P2P structure}
\def\key{\textbf{KEY}}
\def\ip{\textbf{IP}}
\def\ipe{\textbf{IP}^*}
\def\PID{\textbf{PID}}
\def\h{{\ove h}}

The P2P architecture is a Distributed Hash Table.
\begin{defn}
$\key$ is the key space, i.e the set of all the keys. \\
$\ip$ is IP space, i.e. the set of all the IPs of the network. In the ipv4 case, we have 
$\ip=ipv4=\pgra{n\in\mathbb{N}\;|\;0\le n\le 2^{32}-1}$, in the ipv6 case we have
$\ip=ipv6=\pgra{n\in\mathbb{N}\;|\;0\le n\le 2^{128}-1}$.\\
Taking an IP $m\in \ip$, we can write it as a compound of gnode IDs of
different levels: $m=m_l.m_{l-1}. \dots . m_1.m_0$, where $l$ is the highest
level\footnote{for example, in ipv4 we have
$m=m_3*256^3+m_2*256^2+m_1*256^1+m_0$}. $m_i$ is the ID of the gnode of level $i$ where $m$ belongs. For more
information, see the topology document \cite{topodoc}.
\end{defn}
\begin{defn}
The Netsukuku network can host up to $2^{16}$ different P2P services. Each
registered service has a unique identification number called PID (P2P ID). The
set of all PIDs is $\PID=\pgra{n\in \mathbb{N}\;|\;0\le n\le 2^{16}-1}$.\\
A node is a \emph{participant} to a P2P service $p\in \PID$ if it exists in the
network and if it announced its partecipation to the service (we'll see later
how).\\
$\ipe$ is the set of all the IPs of the participant nodes of the network, i.e. 
$\forall x\in \ipe\;\;\exists_1 \T{ a participant node X:\;ip(X)=x}$.
\end{defn}
\begin{defn}
The function $h:\key\longrightarrow \ip$ maps a key $k$ to an IP $x$.
If the keys have the same bit length of the IPs, then h can be
simply defined as the identity function, for example, if $\key$ is the md5
hashes set and $\ip=ipv6$\footnote{An md5 hash has 128 bit, an ipv6 IP too}.\\
The function $H(x):\ip\longrightarrow \ipe$, is defined as follow
\eal{&H(x)=\max \pgra{y \in \ipe\;|\; \forall t\in \ipe\;\;|y-x| \le |y-t| } }
in simple words, $H(x)$ is the closest existent IP to $x$. 
\\
The function $\h:\key \longrightarrow \ipe$ is $\h(k)=H(h(k))$.\\
The node having the IP $\h(k)$, is called hash node of $k$.
\end{defn}

\subsection{Becoming a participant}
A node $g$, in order to become an active participant of a $p\in \PID$, sends a
CTP inside its gnode $G$. This CTP is considered interesting by a node $g'\in
G$, if $g'$ didn't know that $g$ is a participant of $p$ or if the CTP is
interesting as described in the QSPN document\cite{qspndoc}.\\
This same procedure is reitered in all the higher levels: $G$ becomes an
active participant of $p$, because it has at least one participant node. $G$
sends a CTP in level 1, informing all the gnodes it is a participant.
Etcetera.\\

Note: at the end of the above procedures, all the nodes of the network will
know what gnode and nodes are participants to $p$.
Note 2: it isn't necessary to become a participant to store or retrieve data
from $p$.

\subsection{Storing information in $p$}
\label{storeinfop}
Suppose a node $n$ wants to store some data $d$ in the distributed P2P service
$p\in \PID$. It will operate as follow:
\begin{enumerate}
	\item It computes the key $k$ from the data $d$.
	\item It computes $m=\h(k)$:
		\begin{enumerate}
			\item $m':=h(k)$
			\item $m:=H(m')$ is computed with the use of the maps
				of $n$ as follow: 
				\eal{
				&m:=\emph{undef}\\
				&\textbf{for }i=0,1,2,\dots,l\\
				&\qq \emph{let }m_{l-i} \T{ be the closest
				participant (g)node ID to } m'_{l-i}\\
				&\qq \T{[the search of $m_{l-i}$ is restricted
				inside the gnode $n_{l-i+1}$]}\\
				&\qq \textbf{if } m_{l-i} \neq  n_{l-i}\\
				&\qq\qq break\\
				&\textbf{if } m=undef\\
				&\qq \T{There are no participants to $p$}
				}
				Notes:
				\begin{enumerate}
					\item below a certain point, some levels
						of $m$ may be left undefined. This happens
						when the break istruction is
						executed.
					\item if $n$ itself is a participant
						node, then $n_{l-i}$ surely is a
						participant.
				\end{enumerate}
		\end{enumerate}
	\item $n$ sends to $m$ the \emph{store request} $r$:
		\begin{enumerate}
			\item If all the levels of $m$ are defined, then $m\in
				n_1$. Thus, $n$ can directly contact $m$.
			\item otherwise, if the last defined level of $m$ is
				$i$ , then $r$ will be routed by the
				ntk nodes to a node $\ove m\in m_i$.\\
				The node $\ove m$ will then recompute $m=\h(k)$.
				This time, more levels will be defined,
				because $\ove m$ is nearer to $m$ and its
				map can give more accurate information
				(regarding $m$).
			\item This same procedure is reiterated: $\ove m$
				forwards $r$ to a node $\ove {\ove m} \in m_j$,
				where $j<i$. This node will compute $m=h(k)$, and so
				on.
			\item Finally, $m$ will receive the request $r$.
		\end{enumerate}
	\item At this point, the $p$-protocol will dictate what $m$ must do
		with $r$ , f.e. $m$ can store the data in its cache.
	\item To ensure redundancy, if $m$ has accepted the request, it will
		send it to $31$ nodes, which have the closest IP to $m$. 
		In this way, when $m$ dies, they will automatically replace it.
\end{enumerate}
$m$, and all the backup nodes, will automatically delete the stored data
after a time limit specified by the $p$-protocol. For this reason, $n$ has to
repeat the storing procedure periodically. This also has the benefit of
restoring dead hash nodes: if $m$ and all the backup nodes dies simultaneusly,
and if $m\notin n_1$, then $n$ won't be able to know their death, however,
with the next periodic storing procedure, a new hash node $m$ will be elected.\\
\newline
Suppose a node $b$ enters in the gnode $m_1$. Let $B$ be the set containing
the hash node $m$ and all its backup nodes. If the IP of $b$ is closer to
$\h(k)$ than at least one node $b' \in B$, then $m$ will send the request $r$
to $b$, in this way $b$ will become the new hash node or will substitute a
backup node.\\
This same procedure is applied to any level, f.e. suppose that a new gnode $g$
joins in $m_2$. If its ID is closer to $\h(k)$ than the ID of $m_2$, then $m$
will send to $g$ the request $r$. In this way, $g$ will substitute $m_1$.\\
An optimisation to this procedure is based on the following observation: 
if a (g)node $g$ joins the network and if its IP is closer to $\h(k)$ than the
actual hash gnode, then all the successive P2P requests will be
routed to $g$. Since $g$ knows it is a new (g)node, it has to simply forward
the received requests to the old hash node, that is $m$, to see if it knows
the answer to the requests. This method has the advantage that there will be
no delay when a new, closer, hash node joins the network.

\section{Examples of P2P over Ntk}
\begin{enumerate}
	\item 
We'll now present as example a simple P2P Bittorrent\cite{bittorrent}
service.\\\newline

Suppose a node $n$ wants to share a file $f$.
$n$ will calculate the key $k$ from the filename of $f$. 
$n$ will send a store request to $\h(k)$, where the data of the
request is the \emph{.torrent} metafile generated from $f$.
The node $\h(k)$, after having accepted the request, will start up a
bittorrent tracker.
Finally, $n$ will connect to the tracker $\h(k)$, becoming a seeder.\\\newline
The node $m$ that wants to download the file, has to simply calculate $k$ and 
connect to the tracker $\h(k)$.
\item As a further example consider ANDNA\cite{andnadoc}. In ANDNA there are
	two different P2P services in one: the service that stores the IPs and
	the service of the counter nodes, that serves to store the number of
	hostnames registered by a node.
\end{enumerate}


%%%%%%%%%%%%%%%%
% Bibliography %
%%%%%%%%%%%%%%%%
\begin{thebibliography}{99}
	\bibitem{ntksite} Netsukuku website:
		\href{http://netsukuku.freaknet.org/}{http://netsukuku.freaknet.org/}
	\bibitem{ntkrfc} Netsukuku RFCs:
		\href{http://lab.dyne.org/Netsukuku\_RFC}{NTK RFCs}
	\bibitem{qspndoc} QSPN document:
		\href{http://netsukuku.freaknet.org/doc/main\_doc/qspn.pdf}{qspn.pdf}
	\bibitem{topodoc} Netsukuku topology document:
		\href{http://netsukuku.freaknet.org/doc/main\_doc/topology.pdf}{topology.pdf}
	\bibitem{bittorrent} Bittorrent:
		\href{http://en.wikipedia.org/wiki/Bittorrent}{Bittorrent}
	\bibitem{andnadoc} ANDNA document:
		\href{http://netsukuku.freaknet.org/doc/main\_doc/andna.pdf}{andna.pdf}
\end{thebibliography}
\newpage

\begin{center}
\verb|^_^|
\end{center}

\end{document}
