%        File: Ntk_p2p_over_ntk.tex
%     Created: Thu Aug 02 07:00 PM 2007 C
% Last Change: Thu Aug 02 07:00 PM 2007 C
%
% This file is part of Netsukuku
% (c) Copyright 2007 Andrea Lo Pumo aka AlpT <alpt@freaknet.org>
%
% This source code is free software; you can redistribute it and/or
% modify it under the terms of the GNU General Public License as published 
% by the Free Software Foundation; either version 2 of the License,
% or (at your option) any later version.
%
% This source code is distributed in the hope that it will be useful,
% but WITHOUT ANY WARRANTY; without even the implied warranty of
% MERCHANTABILITY or FITNESS FOR A PARTICULAR PURPOSE.
% Please refer to the GNU Public License for more details.
%
% You should have received a copy of the GNU Public License along with
% this source code; if not, write to:
% Free Software Foundation, Inc., 675 Mass Ave, Cambridge, MA 02139, USA.
%

\documentclass[a4paper]{article}
\usepackage{color,graphicx}
\usepackage{amsmath}
\usepackage{amsthm}
\usepackage{amssymb}
\usepackage{amsfonts}
\RequirePackage{ifpdf} % running on pdfTeX?
\ifpdf
\usepackage[pdftex,bookmarks=true,
		   bookmarksnumbered=false,
		   bookmarksopen=false,
		   colorlinks=true,
		   linkcolor=webred] {hyperref}
\definecolor{webgreen}{rgb}{0, 0.5, 0} % less intense green
\definecolor{webblue}{rgb}{0, 0, 0.5} % less intense blue
\definecolor{webred}{rgb}{0.5, 0, 0}   % less intense red
\else
\newcommand{\href}[2]{ #1 }
\fi

%%%% Misc
\newcommand{\T}[1]{\textrm{#1}}
\newcommand{\see}[1]{\T{[\ref{#1},pg.\pageref{#1}]}}
\newcommand{\vedi}[1]{\T{vedi \see{#1}}}
\newcommand{\pgra}[1]{\left\{#1\right\}}
\newcommand{\pqua}[1]{\left[#1\right]}
\newcommand{\pton}[1]{\left(#1\right)}
\newcommand{\pass}[1]{\Big|#1\Big|}
\newcommand{\sist}[1]{{ \begin{cases} #1 \end{cases} }}
\newcommand{\eal}[1]{{\begin{align*} #1 \end{align*}}}
\def\ove#1{{\overline{#1}}}
\newcommand{\qq}{\qquad}
%% Defs
\def\*{{\times}}
\def\|{{\;\lor\;}}
\def\&{{\;\land\;}}
\def\-{{\setminus}}
\def\0{{\emptyset}}
\def\8{{\infty}}
\def\v{{\cup}}
\def\^{{\cap}}
\def\<{{\;\Leftarrow\;}}
\def\>{{\;\Rightarrow\;}}
\def\={{\;\Leftrightarrow\;}}
\def\({{\subseteq}}
\def\){{\supseteq}}
\def\'{{\;\;\;}}
\def\,{{,\;}}

\theoremstyle{definition}
\newtheorem{defn}{Definition}[section]

\title{P2P over Netsukuku\\\small{NTK RFC 0014}}
\author{http://netsukuku.freaknet.org}
\begin{document}
\maketitle
\begin{abstract}
This text describes how is it possible to create a distributed P2P service
over the Netsukuku network.
\end{abstract}

\section{Introduction}

Netsukuku is a distributed, collaborative network of nodes. 
For this reason, the development of P2P applications over Netsukuku is
rather easy.

A P2P application over Ntk can directly access the information
regarding every part of the network by reading the maps and can known
immediately its dynamic changes by listening to QSPN packets.
In order to ease the development of such applications, a ``ntkp2p'' library will
be developed.

\section{P2P structure}
\def\key{\textbf{KEY}}
\def\ip{\textbf{IP}}
\def\ipe{\textbf{IP}^*}
\def\PID{\textbf{PID}}
The P2P architecture is a Distributed Hash Table.
\begin{defn}\qq\\
$\key$ is the key space, i.e the set of all the keys. \\
$\ip$ is IP space, i.e. the set of all the IPs of the network. In the ipv4 case, we have 
$\ip=ipv4=\pgra{n\in\mathbb{N}\;|\;0\le n\le 2^{32}-1}$, in the ipv6 case we have
$\ip=ipv6=\pgra{n\in\mathbb{N}\;|\;0\le n\le 2^{128}-1}$.\\
The Netsukuku network can host up to $2^{16}$ different P2P services. Each
registered service has a unique identification number called PID (P2P ID). The
set of all PIDs is $\PID=\pgra{n\in \mathbb{N}\;|\;0\le n\le 2^{16}-1}$.\\
A node is a \emph{partecipant} to a P2P service $p\in \PID$ if it exists in the
network and if it announced its partecipation to the service (we'll see later
how).\\
$\ipe$ is the set of all the IPs of the partecipant nodes of the network, i.e. 
$\forall x\in \ipe\;\;\exists_1 \T{ a partecipant node X:\;ip(X)=x}$.
\end{defn}
\begin{defn}
The function $h:\key\longrightarrow \ip$ maps a key $k$ to an IP $x$.
If the keys have the same bit length of the IPs, then h can be
simply defined as the identity function, for example, if $\key$ is the md5
hashes set and $\ip=ipv6$\footnote{An md5 hash has 128 bit, an ipv6 IP too}.\\
The function $H(x):\ip\longrightarrow \ipe$, is defined as follow
\eal{&H(x)=\max \pgra{y\in \ipe \;|\; \forall t\in \ipe\;\;|y-x| \le  |y-t| } }
in simple words, $H(x)$ is the closest existent IP to $x$.
\end{defn}

\subsection{Becoming a partecipant}
A node $g$, in order to become an active partecipant of a $p\in \PID$, sends a
CTP inside its gnode $G$. This CTP is considered interesting by a node $g'\in
G$, if $g'$ didn't know that $g$ is a partecipant of $p$ or if the CTP is
interesting as described in the QSPN document\cite{qspndoc}.\\
This same procedure is reitered in all the higher levels: $G$ becomes an
active partecipant of $p$, because it has at least one partecipant node. $G$
sends a CTP in level 1, informing all the gnodes it is a partecipant.
Etcetera.\\

Note: at the end of the above procedures, all the nodes of the network will
know what gnode and nodes are partecipants to $p$.

\subsection{Storing information in $p$}


%%%%%%%%%%%%%%%%
% Bibliography %
%%%%%%%%%%%%%%%%
\begin{thebibliography}{99}
	\bibitem{qspndoc} QSPN document:
		\href{http://netsukuku.freaknet.org/doc/main\_doc/qspn.pdf}{qspn.pdf}
	\bibitem{ntksite} Netsukuku website:
		\href{http://netsukuku.freaknet.org/}{http://netsukuku.freaknet.org/}
\end{thebibliography}
\newpage

\begin{center}
\verb|^_^|
\end{center}

\end{document}
