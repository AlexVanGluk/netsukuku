%        File: inetntk.tex
%     Created: Mon Nov 20 11:00 PM 2006 C
% Last Change: Mon Nov 20 11:00 PM 2006 C
%
% This file is part of Netsukuku
% (c) Copyright 2007 Andrea Lo Pumo aka AlpT <alpt@freaknet.org>
%
% This source code is free software; you can redistribute it and/or
% modify it under the terms of the GNU General Public License as published 
% by the Free Software Foundation; either version 2 of the License,
% or (at your option) any later version.
%
% This source code is distributed in the hope that it will be useful,
% but WITHOUT ANY WARRANTY; without even the implied warranty of
% MERCHANTABILITY or FITNESS FOR A PARTICULAR PURPOSE.
% Please refer to the GNU Public License for more details.
%
% You should have received a copy of the GNU Public License along with
% this source code; if not, write to:
% Free Software Foundation, Inc., 675 Mass Ave, Cambridge, MA 02139, USA.
%

\documentclass[a4paper]{article}
\usepackage{color,graphicx}
\usepackage{amsmath}
\usepackage{amsthm}
\usepackage{amssymb}
\usepackage{amsfonts}
\RequirePackage{ifpdf} % running on pdfTeX?
\ifpdf
\usepackage[pdftex]{hyperref}
\else
\newcommand{\href}[2]{ #1 }
\fi
\title{Internet and Netsukuku}
\author{http://netsukuku.freaknet.org\\AlpT (@freaknet.org)}
\begin{document}
\maketitle

\begin{abstract}
	Netsukuku is completely independent from the Internet. However, a good
	level of integration between the two networks must be preserved.
	In this document, we present the techniques that allow Netsukuku to be
	compatible with the standard Internet. Moreover, we show how it is
	possible to use virtual Internet tunnels to temporarily replace missing
	physical links.
\end{abstract}

\section{Internet compatibility}
Even if Netsukuku has the potential to be a full substitute of the Internet,
it must remain compatible with it for several reasons:
\begin{enumerate}
	\item During the transition phase, the Internet will play a key role
		for the distribution of Netsukuku.
	\item Internet virtual tunnels will be used to temporarily replace
		missing physical links between nodes or entire gnodes.
	\item The Internet will continue to be active and many people will
		rely on it. One optimist prospective is that Netsukuku will
		take its current place, and the Internet will become more and
		more a commercial network, f.e. a huge interactive network-TV.
\end{enumerate}
There are two techniques which ensure the Internet compatibility.
The first restricts the IP assigned to the Netsukuku network to a large
private class. The second is a Linux hack which permits Netsukuku to utilise
the complete set of IPs.\\
Currently only the first has been implemented, however, when the second will
replace it.

\subsection{IP restriction}

Netsukuku, in the Internet compatibility mode, is restricted to a subclass of
ip, so that it doesn't interfere with the public classes of the Internet.
We use the private class A (10.0.0.0) for the ipv4 and the Site-Local class for the
ipv6.\\
The other private classes are not influenced, to let the user create a LAN
with just one gw/node Netsukuku.

You can read more information about the Netsukuku restricted mode, in the
NTK\_RFC 008 \cite{restrictedip}.

\subsection{Net split}
Net Split is a method, which gives Netsukuku the ability to use all the IP
addresses available for a specific Internet Protocol while being compatible
with it.\\
In other words, Netsukuku can use all the ipv4 addresses while avoiding any IP
conflict with the Internet.

For more information about this method, read Net Split \cite{netsplit}.

\section{Internet sharing}
If the nodes are in restricted mode, they can share their Internet connection.
Netsukuku will distribute efficiently the shared connections among the nodes,
in this way every node will automatically know its nearest Internet gateway.

For more information read IGS\cite{IGS}.

\subsection{Distributed Internet connections}
Netsukuku supports a routing method called "multi inet gateway".
A node $n$ can connect to the Internet using, at the same time, multiple
nodes which are sharing their connection.\\
For example, if there are 5 nodes which share their 640Kb/s connections, the
node $n$ will be able to use 5 parallel downloads at 640Kb/s.

Furthermore, even the nodes which share their connections are able to
use the Internet connections shared by the other nodes. In this way, a node
donates its bandwidth but, at the same time, it receives donations from other
users.

For more information read IGS\cite{IGS}.

\section{Virtual to Physical Layer Mapper}
Viphilama stands for \textrm{Virtual to Physical Layer Mapper}.\\
The basic idea of Viphilama is to connect, with Internet tunnels, nodes which
aren't physically linked.
Then whenever, Viphilama finds that a virtual link can be replaced by a
physical one, it removes the virtual link.

Viphilama  will permit to Netsukuku to expand itself over the Internet and then
switch automatically to the physical layer without interfering with the
stability of the Net.

Viphilama transforms Netsukuku into a hybrid overlay network which expands the
original structure of the Internet. Its main advantages are:

\begin{enumerate}
	\item the faster diffusion of Netsukuku: every user with an Internet
		connection can join Netsukuku

	\item the creation of a scalable network which is built upon the Internet
		but is completely separated from it.

	\item the automatic switch from the Netsukuku overlay network to the physical one

	\item the freely registration of domain names (see ANDNA\cite{ANDNA}).

	\item the usage of Carciofo \cite{carciofo} over the Internet.

	\item the workaround of NAT restrictions: even with only one Internet
		connection it is possible to connect an entire LAN to Viphilama. 
		Inside Viphilama, every node of the LAN will get an unique IP,
		therefore the NAT restriction imposed by the ISP (if you want more
		IPs you have to pay) is ignored.
\end{enumerate}



%%%%%%%%%%%%%%%%
% Bibliography %
%%%%%%%%%%%%%%%%

\begin{thebibliography}{99}
	\bibitem{ntksite} Netsukuku website:
		\href{http://netsukuku.freaknet.org/}{http://netsukuku.freaknet.org/}
	\bibitem{ntktopology} Netsukuku topology document:
		\href{http://netsukuku.freaknet.org/doc/main\_doc/topology.pdf}{topology.pdf}
	\bibitem{viphilama} Viphilama NTK\_RFC:
		\href{http://lab.dyne.org/Ntk\_viphilama}{Viphilama}
	\bibitem{IGS} IGS NTK\_RFC:
		\href{http://lab.dyne.org/Ntk\_IGS}{IGS}
	\bibitem{restrictedip} NTK\_RFC 008:
		\href{http://lab.dyne.org/Ntk\_restricted\_ip\_classes}{Restricted ip classes}
	\bibitem{netsplit} Net Split NTK\_RFC:
		\href{http://lab.dyne.org/Ntk\_net\_split}{Net split}
	\bibitem{ANDNA} ANDNA document:
		\href{http://netsukuku.freaknet.org/doc/main\_doc/andna.pdf}{andna.pdf}
	\bibitem{carciofo} Carciofo NTK\_RFC:
		\href{http://lab.dyne.org/Ntk\_carciofo}{Carciofo}
\end{thebibliography}
\newpage

\begin{center}
\verb|^_^|
\end{center}



\end{document}
